\documentclass[12pt]{article}

% Davids Spaghetti Code starts here :)
\usepackage[dvipsnames,usenames, table]{xcolor} 
\usepackage[utf8]{inputenc}
\usepackage[T1]{fontenc}
\usepackage[english]{babel}
\usepackage[a4paper, margin=1in]{geometry}
\usepackage{lipsum}
\usepackage{libertine}
\usepackage{amsfonts,amsmath,amssymb,amsthm} 
%\usepackage{libertinust1math}
\usepackage[footnotes,definitionLists,hashEnumerators,smartEllipses,hybrid]{markdown}
\usepackage{marvosym}
\usepackage{float} 
\usepackage{mathtools}
\DeclarePairedDelimiter{\ceil}{\lceil}{\rceil}
\usepackage{fancyhdr}
\usepackage{booktabs} % for "\midrule" macro
\usepackage{tikz}
\newcommand*\circled[1]{\tikz[baseline=(char.base)]{
            \node[shape=circle,draw,inner sep=2pt] (char) {#1};}}
\usepackage{wasysym} % for smileys
\usepackage{graphicx}
\usepackage{microtype}
\usepackage[hidelinks]{hyperref}
\usepackage{tocloft}
\usepackage{lettrine}
\usepackage{GoudyIn}
\usepackage{forest}
\usepackage{parskip}
% font stuff
\renewcommand{\LettrineFontHook}{\GoudyInfamily{}}
\newcommand{\DECORAR}[3][]{\lettrine[lines=3,loversize=.115,#1]{#2}{#3}}
\def\contentsname{\empty}


% in case we want a TOC
\newcommand{\tocr}{
\renewcommand\cftaftertoctitle{\par\noindent\hrulefill\par\vskip-0.65em}
\tableofcontents
\noindent\hrulefill
}
% some often used commands
\newcommand{\N}{\mathbb{N}}
\newcommand{\Z}{\mathbb{Z}}
\newcommand{\Q}{\mathbb{Q}}

%BEGIN LISTINGDEF
\usepackage{listings}


\renewcommand{\ttdefault}{pcr}

\definecolor{listing-background}{rgb}{0.97,0.97,0.97}
\definecolor{listing-rule}{HTML}{B3B2B3}
\definecolor{listing-numbers}{HTML}{B3B2B3}
\definecolor{listing-text-color}{HTML}{000000}
\definecolor{listing-keyword}{HTML}{435489}
\definecolor{listing-identifier}{HTML}{435489}
\definecolor{listing-string}{HTML}{00999a}
\definecolor{listing-comment}{HTML}{8e8e8e}
\definecolor{listing-javadoc-comment}{HTML}{006CA9}

\lstdefinestyle{eisvogellistingstyle}{
	language=java,
	numbers=left,
	backgroundcolor=\color{listing-background},
	basicstyle=\color{listing-text-color}\small\ttfamily{}, % print whole listing small
	xleftmargin=0.8em, % 2.8 with line numbers
	breaklines=true,
	frame=single,
	framesep=0.6mm,
	rulecolor=\color{listing-rule},
	frameround=ffff,
	framexleftmargin=0.4em, % 2.4 with line numbers | 0.4 without them
	tabsize=4, %width of tabs
	numberstyle=\color{listing-numbers},
	aboveskip=1.0em,
	keywordstyle=\color{listing-keyword}\bfseries, % underlined bold black keywords
	classoffset=0,
	sensitive=true,
	identifierstyle=\color{listing-identifier}, % nothing happens
	commentstyle=\color{listing-comment}, % white comments
	morecomment=[s][\color{listing-javadoc-comment}]{/**}{*/},
	stringstyle=\color{listing-string}, % typewriter type for strings
	showstringspaces=false, % no special string spaces
	escapeinside={/*@}{@*/}, % for comments
	literate=
	{á}{{\'a}}1 {é}{{\'e}}1 {í}{{\'i}}1 {ó}{{\'o}}1 {ú}{{\'u}}1
	{Á}{{\'A}}1 {É}{{\'E}}1 {Í}{{\'I}}1 {Ó}{{\'O}}1 {Ú}{{\'U}}1
	{à}{{\`a}}1 {è}{{\'e}}1 {ì}{{\`i}}1 {ò}{{\`o}}1 {ù}{{\`u}}1
	{À}{{\`A}}1 {È}{{\'E}}1 {Ì}{{\`I}}1 {Ò}{{\`O}}1 {Ù}{{\`U}}1
	{ä}{{\"a}}1 {ë}{{\"e}}1 {ï}{{\"i}}1 {ö}{{\"o}}1 {ü}{{\"u}}1
	{Ä}{{\"A}}1 {Ë}{{\"E}}1 {Ï}{{\"I}}1 {Ö}{{\"O}}1 {Ü}{{\"U}}1
	{â}{{\^a}}1 {ê}{{\^e}}1 {î}{{\^i}}1 {ô}{{\^o}}1 {û}{{\^u}}1
	{Â}{{\^A}}1 {Ê}{{\^E}}1 {Î}{{\^I}}1 {Ô}{{\^O}}1 {Û}{{\^U}}1
	{œ}{{\oe}}1 {Œ}{{\OE}}1 {æ}{{\ae}}1 {Æ}{{\AE}}1 {ß}{{\ss}}1
	{ç}{{\c c}}1 {Ç}{{\c C}}1 {ø}{{\o}}1 {å}{{\r a}}1 {Å}{{\r A}}1
	{€}{{\EUR}}1 {£}{{\pounds}}1 {«}{{\guillemotleft}}1
	{»}{{\guillemotright}}1 {ñ}{{\~n}}1 {Ñ}{{\~N}}1 {¿}{{?`}}1
}
\lstset{style=eisvogellistingstyle}
%END LISTINGDEF


% for hasse diagrams
\usepackage{tikz}
\usetikzlibrary{trees}



% Davids Spaghetti Code ends here

% variables we are setting
\newcommand{\titleVar}{Discrete Mathematics - Exercise 14}
\newcommand{\authorVar}{David Zollikofer}
\newcommand{\es}{\varnothing}
\renewcommand{\tilde}{\sim}
\title{\titleVar}
\date{\today}
\author{\authorVar}
\usepackage{tikz}
\usetikzlibrary{calc,fit,trees}

%making nice headers & footers
\pagestyle{fancy}
\fancyhf{}
\rhead{\titleVar}
\lhead{\authorVar}
\rfoot{Page \thepage}

\begin{document}


\maketitle
%\tocr
\thispagestyle{fancy}

\section*{14.1 Formulas and Statements}
\begin{enumerate}
    \item[a)] a formula
    \item[b)] a statement that is true $\rightarrow$ we have proven this two weeks ago
    \item[c)] intuitively correct but not sure if syntax. It is correct statement about formulas
    \item[d)] a statement about formulas which is true. Again, like we have seen two weeks ago; if $P(x)$ is true no matter the $x$ we can assume that $P(a)$ for a certain $a$ is also true due to universal instantiation.
\end{enumerate}

\section*{14.2 Calculi}
\begin{itemize}
    \item[a)] I claim that rules $R_3$ and $R_5$ are wrong. All the others can easily be checked using a logic table.
    \begin{itemize}
        \item[$R_3$] Assume we have an interpretation $\mathcal{A}$ where $F = 1$ and $G=0$, then $\mathcal{A}(\neg(F\land G)) = 1$, however $\mathcal{A}{(\neg F \land \neg G)} = 0$. Hence this rule is not sound.
        \item[$R_5$] Assume we have an interpretation $\mathcal{A}$ where $F = 0$ and $G=1$, then $\mathcal{A}(F \rightarrow  G) = 1$, however $\mathcal{A}(\neg F \rightarrow \neg G) = 0$. Hence this rule is not sound.
    \end{itemize}
    \item[b)] We have $$\{(D \land A) \rightarrow C, A \land B, B \land A, (B \lor C) \rightarrow D\}$$ as a first step we will get rid of the duplicate $B \land A$ which we can do since we are operating on a set: $$\{(D \land A) \rightarrow C, A \land B, (B \lor C) \rightarrow D\}$$ No we can use the logic calculus $\mathcal{K}' := \{R_1,R_2,R_4,R_6\}$ to derive $A\land B \land C \land D$ from the set of formulas above:
    \begin{equation*}
    \begin{gathered}
        \{A\land B\} \vdash_{R_6} \{A\}\\
        \{A\land B\} \vdash_{R_6} \{B\}\\
        \{B\} \vdash_{R_1} \{B \lor C\}\\
        \{B \lor C, (B \lor C) \rightarrow D\} \vdash_{R_4} \{D\}\\
        \{A,D\} \vdash_{R_6} \{A \land D\}\\
        \{A\land D , (D \land A) \rightarrow C\} \vdash_{R_4} \{C\}\\
        \{A \land B, A \land D\} \vdash_{R_6} \{A \land B \land D\}\\
        \{A \land B \land D , C\} \vdash_{R_6} \{A \land B \land C \land D\}
        \end{gathered}
    \end{equation*}
    
    \item[c)] I claim that it is not complete. assume we have \{A,B\} and we want to show using $\mathcal{K}'$ that $A \lor B$ holds. In the rules above even if we use the implicit $\lor$ hidden in the $\rightarrow$ of rule $R_4$ we would never be able to get rid of the $\neg$ in rule $R_4$ since we do not have any rules for negation. Therefore by example we have shown that $\mathcal{K'}$ is not complete.
    
    \item[d)] A calculus that is complete yet not sound would be the following $$K' = \{\text{all sound statements in propositional logic}\} \cup \{ A \vdash_{ R_{bs}} \neg A\}$$
    
    
\end{itemize}

\section*{14.3 Resolution}
\begin{itemize}
    \item[a)]
        \begin{enumerate}
            \item[i)]We first transform our formula $F \equiv (A\lor B) \land (\neg E) \land (\neg B  \lor D) \land (\neg D \lor E) \land (\neg A \lor B)$ into a set of clauses $K = \{\{A, B\},\{\neg E\},\{\neg B, D\},\{\neg D,E\},\{\neg A , B\}\}$ Now we can use resolution calculus to see that it is unsatisfiable by deriving the empty set from our set of clauses. \par
            \begin{forest}
for tree={
grow'=90,
parent anchor=north,
math content,
before typesetting nodes={
  if level=0{}{
    if content={}{
      shape=coordinate
    }{
      content/.wrap value={\{#1\}},
    },
  },
}
}
[{\varnothing}
    [[{B}
    [{A,B}]
    [{\neg A, B}]
    ]]
    [{\neg B}
    [{\neg D}
    [{\neg E}]
    [{\neg D, E}]
    ]
    [[{\neg B ,D}]]
    ]
]
\end{forest}

            \item[ii)] Instead of showing that it is a tautology we show that its negation is not satisfiable. Negating $G$ gives us the following set of clauses: \\
            $K = \{ \{B,C,\neg D\}, \{B,D\},\{\neg C,\neg D\}, \{\neg B\} \}$ Now we can use resolution calculus to show that $\neg G$ is unsatisfiable.
            \par
            \begin{forest}
for tree={
grow'=90,
parent anchor=north,
math content,
before typesetting nodes={
  if level=0{}{
    if content={}{
      shape=coordinate
    }{
      content/.wrap value={\{#1\}},
    },
  },
}
}
[{\varnothing}
[{B}
[{B,\neg D}
[{B,C,\neg D}]
[{\neg C, \neg D}]
]
[[{B,D}]]
]
[
[[{\neg B}]]]
]
\end{forest}
            \item[iii)] Thanks to lemma 6.3 we can instead show that $M \cup \{\neg H\}$ is not satisfiable: We note that $\neg H \equiv \neg A, \neg C$. Therefore we have the following set of clauses: $K = \{\{\neg A , C\}, \{\neg B, A\},\{A,B\},\{\neg A,\neg C\}\}$. Now we can use resolution calculus to derive the empty set to show unsatisfiability.
            \par
            \begin{forest}
for tree={
grow'=90,
parent anchor=north,
math content,
before typesetting nodes={
  if level=0{}{
    if content={}{
      shape=coordinate
    }{
      content/.wrap value={\{#1\}},
    },
  },
}
}
[{\varnothing}
[{\neg A}
[{\neg A,C}]
[{\neg A, \neg C}]
]
[{A}
[{\neg B,A}]
[{A,B}]
]
]
\end{forest}
            
            
        \end{enumerate}
    \item[b)] Assume we have $n$ symbols in our set of clauses. This can be assumed since the clauses are finite. We will show that the set of all clauses is finite. Assume that we only have one symbol $A_1$. Then there are 4 clauses, namely: $\{\varnothing, \{A_1\},\{\neg A_1\},\{A_1,\neg A_1\}\}$. This is smaller than $4^n$. We will inductively show that for $n$ symbols we always have less than $4^n$ possible clauses. assume we have $4^n$ clauses and we now add a new symbol $A_{n+1}$. How many options are there? We will have to look at every clause $K_i$ and have four options: \begin{enumerate}
        \item $K_i' = K_i \cup \varnothing$
        \item $K_i' = K_i \cup A_{n+1}$
        \item $K_i' = K_i \cup \neg A_{n+1}$
        \item $K_i' = K_i \cup \{A_{n+1}, \neg A_{n+1}\}$
    \end{enumerate}
    So for every clause there are four combinations. Assuming we had $4^n$ clauses and every clause can be changed in 4 ways we have $4^n * 4 =4^{n+1}$ clauses which concludes the inductive step. This also shows that the set of clauses is always smaller than $4^n$ where $n$ denotes the number of symbols in our clauses.
    \item[c)] Not sure how to solve this one. The hint on the exercise sheet is a bit cryptic.
\end{itemize}



\end{document}