\documentclass[12pt]{article}


\usepackage[utf8]{inputenc}
\usepackage[english]{babel}


\usepackage{amsmath,amssymb,amsfonts,mathrsfs,xcolor,url}
\usepackage{graphicx}
\graphicspath{{./img/}}


\usepackage{parskip}


\usepackage{algorithm}
\usepackage{algorithmic}


\usepackage{amsthm}
\usepackage{thmtools}




\newtheorem{theorem}{Theorem}[section]
\newtheorem{lemma}[theorem]{Lemma}

\theoremstyle{definition}

\declaretheorem[style=definition,name=Definition,qed=$\lrcorner$]{definition}



\declaretheorem[style=definition,name=Example,qed=$\lozenge$]{example}


\theoremstyle{remark}
\newtheorem*{remark}{Remark}

\usepackage{todonotes}
\usepackage{beton}


\usepackage{concmath}

\DeclareFontSeriesDefault[rm]{bf}{sbc}
\usepackage[a4paper, margin=1.2in]{geometry}
\usepackage{mdframed}

% begin custom commands
\newcommand{\Q}{\mathbb{Q}}
\newcommand{\R}{\mathbb{R}}
\newcommand{\N}{\mathbb{N}}
\newcommand{\E}{\mathbb{E}}
\renewcommand{\O}[1]{\mathcal{O}\left(#1\right)}
\newcommand{\C}{\mathcal{C}}
\newcommand{\trs}{\operatorname{trs}}
\usepackage{fancyhdr}

\pagestyle{fancy}
\fancyhf{}
\fancyhead[L]{Lecture Name}
\fancyhead[R]{Assignment Name}
\fancyfoot[C]{Page \thepage}


	


\begin{document}
	\thispagestyle{fancy}
\begin{center}
	{\Huge Assignment Name}  \\ Author Name\\
\today
\end{center}

\hrule
%%%%%%%%%%%%%%%%%%%%%%%%%%%%%%%%%%%%%%%%%%%%%%%%%%%%%%%%%%%%%%%%%%%%%%%%

\vspace*{5mm}

\section{Task}
For a given fixed convex polygon $\C$ with vertices $p_0,\ldots p_{n-1}$ in counterclockwise order and a non-vertical query line $q$, return the area $A_q$ in $\C$ under $q$ in time $\O{\log n}$.


\section{Solution}

\paragraph{Idea}

Let us first sketch how we will solve this algorithmic problem efficiently. All the following definitions can be seen in figure \ref{fig:approach} which greatly helps understanding the algorithm.
We will follow a two step approach.

First, we will have to find out where $q$ intersects the convex polygon. There are three different options for this:
\begin{enumerate}
\item The line $q$ is completely above $\C$. In this case we will return the whole area of $\C$. The area calculation will be covered in part two.
\item The line $q$ is completely below $\C$. In this case we will return the an area of 0.
\item The line $q$ intersects $\C$. This will give us two points where the convex polygon $\C$ and line $q$ intersect (one where $q$ ``enters" $\C$ and one where $q$ ``exits" $\C$). By assumption $q$ is not a vertical line giving the intersection points two different $x$-coordinates. Let the intersection point with the smaller $x$-coordinate be denoted by $i_l$ (for intersection-left) and the one with the bigger $x$-coordinate $i_r$ (for intersection-right).
\end{enumerate}

Using an idea very similar to the \textit{Reporting Points Below a Query Line Problem} as seen in the lecture we will identify the two vertices $p_l$ and $p_r$. We define $p_l$ ($p_r$) as the point $p_i$ in the set of points which make up $\C$ which is the lower vertex of the edge which $q$ has pierced while intersecting $\C$, or in other words, the vertex of smaller $y$-coordinate of the edge of $\C$ on which $i_l$ ($i_r$) is located on. Note that $p_l$ and $p_r$ can also refer to the same point.

Having located $i_l, i_r, p_l, p_r$ as shown in figure \ref{fig:approach} we can move on to finding the area.

%\begin{figure}[H]
%	\centering
%	\includegraphics[width=0.5\textwidth]{generalApproach.png}
%	\caption{Sketch of the algorithm for finding the grey area of figure \ref{fig:sit}. Note that by construction $A_{q,a}\cup A_{q,f} = A_{q}$}
%	\label{fig:approach}
%\end{figure}


Let $A_q$ be the are we are looking for. To find $A_q$ efficiently we split it up into two different subareas $A_{q,a}$ and $A_{q,f}$ with $A_{q,a}\cup A_{q,f} = A_{q}$.

\begin{itemize}
	\item $A_{q,f}$ (fixed area under $q$): The area of the convex polygon formed by including all vertices of $\C$ starting at $p_l$ and walking along $\C$ in a counter-clockwise rotation and ending with $p_r$ (orange in figure \ref{fig:approach}).
	\item $A_{q,a}$ (adaptive area under $q$): The area of the polygon formed by the points $\{p_l,p_r,i_r,i_l\}$ (blue in figure \ref{fig:approach}). Note that we can find this area by splitting it up into two triangles (see dashed line in figure \ref{fig:approach}) and calculating the area of the two individual triangles.
\end{itemize}


Once we have found the values for $A_{q,f}$ and $A_{q,a}$ we return their sum as the answer. 


Having covered the basic idea behind the algorithm we can now develop the algorithm rigorously.

\bigskip


\begin{mdframed}
	\textbf{Important Note \quad} To make indexing easier we will adopt modular indexing in this paper. This means all indexes $p_i$ should be read as $$p_i = p_{ j} \text{ \quad where \quad } j=i\mod (\text{length of array}).$$ As an example consider $[p_0,p_1,p_2]$, then the point $p_{i+1}$ for $i=2$ is $p_0$.
	
	From this point on the notion of modular indexing is implicitly assumed.
\end{mdframed}





\begin{thebibliography}{9}
	\bibitem{lecnotes} 
 	Some entry here
	
\end{thebibliography}


\end{document}
