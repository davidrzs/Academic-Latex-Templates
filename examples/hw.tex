\documentclass[12pt]{article}


\usepackage[dvipsnames,usenames, table]{xcolor} 
\usepackage[utf8]{inputenc}
\usepackage[T1]{fontenc}
\usepackage[english]{babel}
\usepackage[a4paper, margin=1in]{geometry}
\usepackage{lipsum}
\usepackage{libertine}
\usepackage{amsfonts,amsmath,amssymb,amsthm} 
\usepackage{eulervm}
\usepackage{marvosym}
\usepackage{float} 
\usepackage{mathtools}
\DeclarePairedDelimiter{\ceil}{\lceil}{\rceil}
\usepackage{fancyhdr}
\usepackage{booktabs} % for "\midrule" macro
\usepackage{tikz}
\usepackage{graphicx}
\usepackage{microtype}
\usepackage[hidelinks]{hyperref}
\usepackage{tocloft}
\usepackage{lettrine}
\usepackage{GoudyIn}
\usepackage{listings}
\usepackage{forest}
\usepackage{parskip}
% font stuff
\renewcommand{\LettrineFontHook}{\GoudyInfamily{}}
\newcommand{\DECORAR}[3][]{\lettrine[lines=3,loversize=.115,#1]{#2}{#3}}
\def\contentsname{\empty}


% in case we want a TOC
\newcommand{\tocr}{
\renewcommand\cftaftertoctitle{\par\noindent\hrulefill\par\vskip-0.65em}
\tableofcontents
\noindent\hrulefill
}
% some often used commands
\newcommand{\N}{\mathbb{N}}
\newcommand{\Z}{\mathbb{Z}}
\newcommand{\Q}{\mathbb{Q}}

% LISTINGDEF START
    

\usepackage{inconsolata}

\definecolor{background}{HTML}{fcfeff}
\definecolor{comment}{HTML}{b2979b}
\definecolor{keywords}{HTML}{255957}
\definecolor{basicStyle}{HTML}{6C7680}
\definecolor{variable}{HTML}{001080}
\definecolor{string}{HTML}{c18100}
\definecolor{numbers}{HTML}{3f334d}
\ifx
\lstset{
	% How/what to match
	sensitive=true,
	% Border (above and below)
	frame=single,
	% Extra margin on line (align with paragraph)
	xleftmargin=\parindent,
	% Put extra space under caption
	belowcaptionskip=1\baselineskip,
	
	% Break long lines into multiple lines?
	breaklines=true,
	% Show a character for spaces?
	showstringspaces=false,
	tabsize=2
}
\fi
%END LISTINGDEF
\lstdefinestyle{lectureNotesListing}{
	numbers=left,
	xleftmargin=0.8em, % 2.8 with line numbers
	breaklines=true,
	frame=single,
	framesep=0.6mm,
	frameround=ffff,
	framexleftmargin=0.4em, % 2.4 with line numbers | 0.4 without them
	tabsize=4, %width of tabs
	aboveskip=1.0em,
	classoffset=0,
	sensitive=true,
	% Colors
	backgroundcolor=\color{background},
	basicstyle=\color{basicStyle}\small\ttfamily,
	keywordstyle=\color{keywords},
	commentstyle=\color{comment},
	stringstyle=\color{string},
	numberstyle=\color{numbers},
	identifierstyle=\color{variable},
	showstringspaces=true
}
\lstset{style=lectureNotesListing}
%END LISTINGDEF
% LISTINGDEF END


% for hasse diagrams
\usepackage{tikz}
\usetikzlibrary{trees}





% variables we are setting
\newcommand{\titleVar}{Homework Template}
\newcommand{\authorVar}{Author Name}
\newcommand{\es}{\varnothing}
\renewcommand{\tilde}{\sim}
\title{\titleVar}
\date{\today}
\author{\authorVar}
\usepackage{tikz}
\usetikzlibrary{calc,fit,trees}

%making nice headers & footers
\pagestyle{fancy}
\fancyhf{}
\rhead{\titleVar}
\lhead{\authorVar}
\rfoot{Page \thepage}

\begin{document}


\maketitle

\thispagestyle{fancy}


\section*{14.2 Calculi}
\begin{itemize}
    \item[b)] We have $$\{(D \land A) \rightarrow C, A \land B, B \land A, (B \lor C) \rightarrow D\}$$and now some mathemagic  $$\{(D \land A) \rightarrow C, A \land B, (B \lor C) \rightarrow D\}$$ Using derivation calculus we get $\mathcal{K}' := \{R_1,R_2,R_4,R_6\}$ to derive $A\land B \land C \land D$ from the set of formulas above:
    \begin{equation*}
    \begin{gathered}
        \{A\land B\} \vdash_{R_6} \{A\}\\
        \{A\land B\} \vdash_{R_6} \{B\}\\
 			\ldots \\
        \{A \land B, A \land D\} \vdash_{R_6} \{A \land B \land D\}\\
        \{A \land B \land D , C\} \vdash_{R_6} \{A \land B \land C \land D\}
        \end{gathered}
    \end{equation*}
    
    \item[c)] I claim that it is not complete. assume we have \{A,B\} and we want to show using $\mathcal{K}'$ that $A \lor B$ holds. In the rules above even if we use the implicit $\lor$ hidden in the $\rightarrow$ of rule $R_4$ we would never be able to get rid of the $\neg$ in rule $R_4$ since we do not have any rules for negation. Therefore by example we have shown that $\mathcal{K'}$ is not complete.
    
\end{itemize}

\section*{Some Code}
Lorem ipsum dolor sit amet, consetetur sadipscing elitr, sed diam nonumy eirmod tempor invidunt ut labore et dolore magna aliquyam erat, sed diam voluptua. At vero eos et accusam et justo duo dolores et ea rebum. Stet clita kasd gubergren, no sea takimata sanctus est Lorem ipsum dolor sit amet.
\begin{lstlisting}[language=Python]
# prints hello world
def main():
a = 12
print("hello world")
\end{lstlisting}
\lipsum[2]



\end{document}